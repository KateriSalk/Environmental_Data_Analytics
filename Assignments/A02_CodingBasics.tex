\documentclass[]{article}
\usepackage{lmodern}
\usepackage{amssymb,amsmath}
\usepackage{ifxetex,ifluatex}
\usepackage{fixltx2e} % provides \textsubscript
\ifnum 0\ifxetex 1\fi\ifluatex 1\fi=0 % if pdftex
  \usepackage[T1]{fontenc}
  \usepackage[utf8]{inputenc}
\else % if luatex or xelatex
  \ifxetex
    \usepackage{mathspec}
  \else
    \usepackage{fontspec}
  \fi
  \defaultfontfeatures{Ligatures=TeX,Scale=MatchLowercase}
\fi
% use upquote if available, for straight quotes in verbatim environments
\IfFileExists{upquote.sty}{\usepackage{upquote}}{}
% use microtype if available
\IfFileExists{microtype.sty}{%
\usepackage{microtype}
\UseMicrotypeSet[protrusion]{basicmath} % disable protrusion for tt fonts
}{}
\usepackage[margin=2.54cm]{geometry}
\usepackage{hyperref}
\hypersetup{unicode=true,
            pdftitle={Assignment 2: Coding Basics},
            pdfauthor={Wanchen Xiong},
            pdfborder={0 0 0},
            breaklinks=true}
\urlstyle{same}  % don't use monospace font for urls
\usepackage{color}
\usepackage{fancyvrb}
\newcommand{\VerbBar}{|}
\newcommand{\VERB}{\Verb[commandchars=\\\{\}]}
\DefineVerbatimEnvironment{Highlighting}{Verbatim}{commandchars=\\\{\}}
% Add ',fontsize=\small' for more characters per line
\usepackage{framed}
\definecolor{shadecolor}{RGB}{248,248,248}
\newenvironment{Shaded}{\begin{snugshade}}{\end{snugshade}}
\newcommand{\KeywordTok}[1]{\textcolor[rgb]{0.13,0.29,0.53}{\textbf{#1}}}
\newcommand{\DataTypeTok}[1]{\textcolor[rgb]{0.13,0.29,0.53}{#1}}
\newcommand{\DecValTok}[1]{\textcolor[rgb]{0.00,0.00,0.81}{#1}}
\newcommand{\BaseNTok}[1]{\textcolor[rgb]{0.00,0.00,0.81}{#1}}
\newcommand{\FloatTok}[1]{\textcolor[rgb]{0.00,0.00,0.81}{#1}}
\newcommand{\ConstantTok}[1]{\textcolor[rgb]{0.00,0.00,0.00}{#1}}
\newcommand{\CharTok}[1]{\textcolor[rgb]{0.31,0.60,0.02}{#1}}
\newcommand{\SpecialCharTok}[1]{\textcolor[rgb]{0.00,0.00,0.00}{#1}}
\newcommand{\StringTok}[1]{\textcolor[rgb]{0.31,0.60,0.02}{#1}}
\newcommand{\VerbatimStringTok}[1]{\textcolor[rgb]{0.31,0.60,0.02}{#1}}
\newcommand{\SpecialStringTok}[1]{\textcolor[rgb]{0.31,0.60,0.02}{#1}}
\newcommand{\ImportTok}[1]{#1}
\newcommand{\CommentTok}[1]{\textcolor[rgb]{0.56,0.35,0.01}{\textit{#1}}}
\newcommand{\DocumentationTok}[1]{\textcolor[rgb]{0.56,0.35,0.01}{\textbf{\textit{#1}}}}
\newcommand{\AnnotationTok}[1]{\textcolor[rgb]{0.56,0.35,0.01}{\textbf{\textit{#1}}}}
\newcommand{\CommentVarTok}[1]{\textcolor[rgb]{0.56,0.35,0.01}{\textbf{\textit{#1}}}}
\newcommand{\OtherTok}[1]{\textcolor[rgb]{0.56,0.35,0.01}{#1}}
\newcommand{\FunctionTok}[1]{\textcolor[rgb]{0.00,0.00,0.00}{#1}}
\newcommand{\VariableTok}[1]{\textcolor[rgb]{0.00,0.00,0.00}{#1}}
\newcommand{\ControlFlowTok}[1]{\textcolor[rgb]{0.13,0.29,0.53}{\textbf{#1}}}
\newcommand{\OperatorTok}[1]{\textcolor[rgb]{0.81,0.36,0.00}{\textbf{#1}}}
\newcommand{\BuiltInTok}[1]{#1}
\newcommand{\ExtensionTok}[1]{#1}
\newcommand{\PreprocessorTok}[1]{\textcolor[rgb]{0.56,0.35,0.01}{\textit{#1}}}
\newcommand{\AttributeTok}[1]{\textcolor[rgb]{0.77,0.63,0.00}{#1}}
\newcommand{\RegionMarkerTok}[1]{#1}
\newcommand{\InformationTok}[1]{\textcolor[rgb]{0.56,0.35,0.01}{\textbf{\textit{#1}}}}
\newcommand{\WarningTok}[1]{\textcolor[rgb]{0.56,0.35,0.01}{\textbf{\textit{#1}}}}
\newcommand{\AlertTok}[1]{\textcolor[rgb]{0.94,0.16,0.16}{#1}}
\newcommand{\ErrorTok}[1]{\textcolor[rgb]{0.64,0.00,0.00}{\textbf{#1}}}
\newcommand{\NormalTok}[1]{#1}
\usepackage{graphicx,grffile}
\makeatletter
\def\maxwidth{\ifdim\Gin@nat@width>\linewidth\linewidth\else\Gin@nat@width\fi}
\def\maxheight{\ifdim\Gin@nat@height>\textheight\textheight\else\Gin@nat@height\fi}
\makeatother
% Scale images if necessary, so that they will not overflow the page
% margins by default, and it is still possible to overwrite the defaults
% using explicit options in \includegraphics[width, height, ...]{}
\setkeys{Gin}{width=\maxwidth,height=\maxheight,keepaspectratio}
\IfFileExists{parskip.sty}{%
\usepackage{parskip}
}{% else
\setlength{\parindent}{0pt}
\setlength{\parskip}{6pt plus 2pt minus 1pt}
}
\setlength{\emergencystretch}{3em}  % prevent overfull lines
\providecommand{\tightlist}{%
  \setlength{\itemsep}{0pt}\setlength{\parskip}{0pt}}
\setcounter{secnumdepth}{0}
% Redefines (sub)paragraphs to behave more like sections
\ifx\paragraph\undefined\else
\let\oldparagraph\paragraph
\renewcommand{\paragraph}[1]{\oldparagraph{#1}\mbox{}}
\fi
\ifx\subparagraph\undefined\else
\let\oldsubparagraph\subparagraph
\renewcommand{\subparagraph}[1]{\oldsubparagraph{#1}\mbox{}}
\fi

%%% Use protect on footnotes to avoid problems with footnotes in titles
\let\rmarkdownfootnote\footnote%
\def\footnote{\protect\rmarkdownfootnote}

%%% Change title format to be more compact
\usepackage{titling}

% Create subtitle command for use in maketitle
\newcommand{\subtitle}[1]{
  \posttitle{
    \begin{center}\large#1\end{center}
    }
}

\setlength{\droptitle}{-2em}

  \title{Assignment 2: Coding Basics}
    \pretitle{\vspace{\droptitle}\centering\huge}
  \posttitle{\par}
    \author{Wanchen Xiong}
    \preauthor{\centering\large\emph}
  \postauthor{\par}
    \date{}
    \predate{}\postdate{}
  

\begin{document}
\maketitle

\subsection{OVERVIEW}\label{overview}

This exercise accompanies the lessons in Environmental Data Analytics
(ENV872L) on coding basics in R.

\subsection{Directions}\label{directions}

\begin{enumerate}
\def\labelenumi{\arabic{enumi}.}
\tightlist
\item
  Change ``Student Name'' on line 3 (above) with your name.
\item
  Use the lesson as a guide. It contains code that can be modified to
  complete the assignment.
\item
  Work through the steps, \textbf{creating code and output} that fulfill
  each instruction.
\item
  Be sure to \textbf{answer the questions} in this assignment document.
  Space for your answers is provided in this document and is indicated
  by the ``\textgreater{}'' character. If you need a second paragraph be
  sure to start the first line with ``\textgreater{}''. You should
  notice that the answer is highlighted in green by RStudio.
\item
  When you have completed the assignment, \textbf{Knit} the text and
  code into a single PDF file. You will need to have the correct
  software installed to do this (see Software Installation Guide) Press
  the \texttt{Knit} button in the RStudio scripting panel. This will
  save the PDF output in your Assignments folder.
\item
  After Knitting, please submit the completed exercise (PDF file) to the
  dropbox in Sakai. Please add your last name into the file name (e.g.,
  ``Salk\_A02\_CodingBasics.pdf'') prior to submission.
\end{enumerate}

The completed exercise is due on Thursday, 24 January, 2019 before class
begins.

\subsection{Basics Day 1}\label{basics-day-1}

\begin{enumerate}
\def\labelenumi{\arabic{enumi}.}
\item
  Generate a sequence of numbers from one to 100, increasing by fours.
  Assign this sequence a name.
\item
  Compute the mean and median of this sequence.
\item
  Ask R to determine whether the mean is greater than the median.
\item
  Insert comments in your code to describe what you are doing.
\end{enumerate}

\begin{Shaded}
\begin{Highlighting}[]
\CommentTok{# produce a sequence of numbers from 1 to 100 increasing by 4, and assign this sequence as sequence1}
\NormalTok{sequence1 <-}\StringTok{ }\KeywordTok{seq}\NormalTok{(}\DecValTok{1}\NormalTok{, }\DecValTok{100}\NormalTok{, }\DecValTok{4}\NormalTok{) }
\CommentTok{# calculate the mean of sequence1}
\KeywordTok{mean}\NormalTok{(sequence1) }
\end{Highlighting}
\end{Shaded}

\begin{verbatim}
## [1] 49
\end{verbatim}

\begin{Shaded}
\begin{Highlighting}[]
\CommentTok{# calculate the median of sequence1}
\KeywordTok{median}\NormalTok{(sequence1)}
\end{Highlighting}
\end{Shaded}

\begin{verbatim}
## [1] 49
\end{verbatim}

\begin{Shaded}
\begin{Highlighting}[]
\CommentTok{# determine whether the mean of the sequence is greater than its median}
\KeywordTok{mean}\NormalTok{(sequence1) }\OperatorTok{>}\StringTok{ }\KeywordTok{median}\NormalTok{(sequence1)}
\end{Highlighting}
\end{Shaded}

\begin{verbatim}
## [1] FALSE
\end{verbatim}

\subsection{Basics Day 2}\label{basics-day-2}

\begin{enumerate}
\def\labelenumi{\arabic{enumi}.}
\setcounter{enumi}{4}
\item
  Create a series of vectors, each with four components, consisting of
  (a) names of students, (b) test scores out of a total 100 points, and
  (c) whether or not they have passed the test (TRUE or FALSE) with a
  passing grade of 50.
\item
  Label each vector with a comment on what type of vector it is.
\item
  Combine each of the vectors into a data frame. Assign the data frame
  an informative name.
\item
  Label the columns of your data frame with informative titles.
\end{enumerate}

\begin{Shaded}
\begin{Highlighting}[]
\CommentTok{#Create a character vector with four student names}
\NormalTok{name <-}\StringTok{ }\KeywordTok{c}\NormalTok{(}\StringTok{"Adam"}\NormalTok{, }\StringTok{"Brandon"}\NormalTok{, }\StringTok{"Charles"}\NormalTok{, }\StringTok{"Daniel"}\NormalTok{)}
\CommentTok{#Create a numerical vector with scores of each student}
\NormalTok{score <-}\StringTok{ }\KeywordTok{c}\NormalTok{(}\DecValTok{89}\NormalTok{, }\DecValTok{88}\NormalTok{, }\DecValTok{50}\NormalTok{, }\DecValTok{49}\NormalTok{)}
\CommentTok{#Create a logical vecror}
\NormalTok{pass <-}\StringTok{ }\NormalTok{score }\OperatorTok{>}\StringTok{ }\DecValTok{50}
\CommentTok{#Create a dataframe that shows all the student and their score information}
\NormalTok{studentinfo <-}\StringTok{ }\KeywordTok{data.frame}\NormalTok{(name, score, pass)}
\end{Highlighting}
\end{Shaded}

\begin{enumerate}
\def\labelenumi{\arabic{enumi}.}
\setcounter{enumi}{8}
\tightlist
\item
  QUESTION: How is this data frame different from a matrix?
\end{enumerate}

\begin{quote}
ANSWER: Matrix contains only one type of data like numbers, whereas data
frame can contain different types of data.
\end{quote}

\begin{enumerate}
\def\labelenumi{\arabic{enumi}.}
\setcounter{enumi}{9}
\item
  Create a function with an if/else statement. Your function should
  determine whether a test score is a passing grade of 50 or above (TRUE
  or FALSE). You will need to choose either the \texttt{if} and
  \texttt{else} statements or the \texttt{ifelse} statement. The name of
  your function should be informative.
\item
  Apply your function to the vector with test scores that you created in
  number 5.
\end{enumerate}

\begin{Shaded}
\begin{Highlighting}[]
\CommentTok{#create a function using ifelse statement, if the score is equal to or more than 50, then returns true, if not, then returns false}
\NormalTok{check_score <-}\StringTok{ }\ControlFlowTok{function}\NormalTok{(x) \{}
  \KeywordTok{ifelse}\NormalTok{(x }\OperatorTok{>=}\StringTok{ }\DecValTok{50}\NormalTok{, }\OtherTok{TRUE}\NormalTok{, }\OtherTok{FALSE}\NormalTok{)}
\NormalTok{\}}

\CommentTok{#Apply this function to the numerical vector and check the results}
\NormalTok{pass_or_not <-}\StringTok{ }\KeywordTok{check_score}\NormalTok{(score) ; pass_or_not}
\end{Highlighting}
\end{Shaded}

\begin{verbatim}
## [1]  TRUE  TRUE  TRUE FALSE
\end{verbatim}

\begin{enumerate}
\def\labelenumi{\arabic{enumi}.}
\setcounter{enumi}{11}
\tightlist
\item
  QUESTION: Which option of \texttt{if} and \texttt{else} vs.
  \texttt{ifelse} worked? Why?
\end{enumerate}

\begin{quote}
ANSWER: ``if'' and ``else'' did not work out because it applies to one
element; however, the vector had more than one elements, thus only the
first element was used. ``ifelse'' worked because it returned value
based on the elements in the condition; it worked with multiple
elements.
\end{quote}


\end{document}
